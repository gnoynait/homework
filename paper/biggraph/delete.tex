

Some graph algorithms need to change the graph's topology.
A clustering algorithm, for example, might replace each
cluster with a single vertex, and a minimum spanning tree
algorithm might remove all but the tree edges.
User defined function can not only send messages, but also can
issue requests to add or remove vertices or edges.


Multiple vertices may issue conflicting requests in the same
superstep. To solve this problem, Pregel use two mechanisms to achieve
determinism:
\begin{itemize}
  \item Partial ordering. As with messages, mutations become elective in the superstep after the requests were
  issued. Within that superstep removals are performed first, with edge removal before vertex removal. Then additions follow, with vertex addition before edge addition. This partial ordering yields deterministic results for most conflicts.
  \item Handlers. The remaining conflicts are resolved by user-defined handlers. For these conflicts, system have a default random resolution. But users with special needs may specify a better conflict resolution policy by defining an appropriate handler method.
\end{itemize}